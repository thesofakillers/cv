%%%%%%%%%%%%%%%%%%%%%%%%%%%%%%%%%%%%%%%%%
% Medium Length Professional CV
% LaTeX Template
% Version 2.0 (8/5/13)
%
% This template has been downloaded from:
% http://www.LaTeXTemplates.com
%
% Original author:
% Rishi Shah
%
% Important note:
% This template requires the resume.cls file to be in the same directory as the
% .tex file. The resume.cls file provides the resume style used for structuring the
% document.
%
%%%%%%%%%%%%%%%%%%%%%%%%%%%%%%%%%%%%%%%%%

%----------------------------------------------------------------------------------------
%	PACKAGES AND OTHER DOCUMENT CONFIGURATIONS
%----------------------------------------------------------------------------------------

\documentclass{resume} % Use the custom resume.cls style
\usepackage{hyperref}
\usepackage{enumitem}
\setlist[itemize]{noitemsep, topsep=0pt}
\usepackage[left=0.5in,top=0.4in,right=0.5in,bottom=0.4in]{geometry} % Document margins
\newcommand{\tab}[1]{\hspace{.2667\textwidth}\rlap{#1}}
\newcommand{\itab}[1]{\hspace{0em}\rlap{#1}}
\name{Giulio Starace} % Your name
\address{giulio.starace@gmail.com \\ \today}

\begin{document}

\begin{rSection}{Education}

	{\bf MSc Artificial Intelligence} \hfill \textbf{Amsterdam, The Netherlands}
	\\\href{https://www.dur.ac.uk/}{\textit{University of Amsterdam}} \hfill Current GPA: 8.32 \hfill {\em September 2021
		- 2023 (ongoing)}
	\begin{itemize}
		\item Current thesis direction: Natural language and Goal Misgeneralization In Offline
		      Reinforcement Learning
		\item Topics of note: Machine Learning, Deep Learning, Natural Language Processing, Information
		      Retrieval,\\ Reinforcement Learning, Spiking Neural Networks
	\end{itemize}

	{\bf ABC Summer School - Computations in Consciousness and Perception} \hfill \textbf{Amsterdam, The Netherlands}
	\\\href{https://www.dur.ac.uk/}{\textit{University of Amsterdam}} \hfill {\em June 2022}
	\begin{itemize}
		\item 2-week full-time computational neuroscience programme. Developed Project proposal
		      available at \href{https://www.giuliostarace.com/ltm/}{giuliostarace.com/ltm}.
	\end{itemize}

	{\bf BSc Hons. in Natural Sciences (Physics \& Computer Science)} \hfill \textbf{Durham, UK}
	\\\href{https://www.dur.ac.uk/}{\textit{Durham University}}\hfill GPA: First-Class Honours \hfill {\em October 2016 - July 2019}
	\begin{itemize}\vspace{-0.5em}
		\item Topics of note: Statistics, Vector Calculus, Linear Algebra, Thermodynamics, Quantum \&
		      Particle Physics, Databases, Algorithms and Data-structures, Cryptography and Information,
		      Compiler Design, Operating Systems, Computer Vision
	\end{itemize}
\end{rSection}

\begin{rSection}{Experience}

	{\bf \href{https://icai.ai/airlab/}{University of Amsterdam - AIRLab}} \hfill \textbf{Amsterdam,
		The Netherlands}
	\\ \textit{Part-time Researcher (Research Assistant)} \hfill {\em April 2022 - October 2022}
	\begin{itemize}\vspace{-0.5em}
		\item Cross-lingual adaptation of large language models (LLMs). Paper and Code:
		      \href{https://github.com/thesofakillers/claficle}{github.com/thesofakillers/claficle}
		\item Proposed novel ``Post-Hoc Disentanglement via Vessel Adapters (PHoDiVA)'' method.
	\end{itemize}

	{\bf \href{https://www.eso.org}{ESO} - European Southern Observatory} \hfill \textbf{Cerro Paranal, Chile}
	\\ \textit{Research Engineer} \hfill {\em September 2020 - September 2021}
	\begin{itemize}\vspace{-0.5em}
		\item High-resolution forecasting of Astronomical Seeing using Machine Learning
		      methods, using ARIMA and LSTM
		\item Led the development of a unified data processing and evaluation
		      pipeline: \href{https://pypi.org/project/nowcastlib/}{pypi.org/project/nowcastlib}
	\end{itemize}
	\textit{Software Engineering Intern} \hfill {\em August 2019 - January 2020}
	\begin{itemize}\vspace{-0.5em}
		\item Designed, developed, tested and deployed a system for displaying information from the
		      observatory's several ticket systems in unified and personalized way via Microsoft Graph API.
	\end{itemize}

	{\bf \href{https://video.io/}{Video.io - startup}} \hfill \textbf{Remote, Anywhere}
	\\ \textit{ML Engineer} \hfill {\em February 2020 - June 2020}
	\begin{itemize}\vspace{-0.5em}
		\item Worked on implementing abstractive summarization of dialogic video. (Pandas, Google Cloud
		      Speech API).
		\item Real time speech to text captioning of live video streams (NodeJS Streams, FFmpeg,
		      HLS/DASH video).
		\item Revamped full-stack of internal experimental website (NodeJS, GraphQL, PostgreSQL, Docker,
		      NGINX, React).
	\end{itemize}

	{\bf \href{https://global.jaxa.jp/}{JAXA} -  Japan Aerospace Exploration Agency} \hfill \textbf{Sagamihara, Japan}
	\\ \textit{Research Intern} \hfill {\em August 2018 - October 2018}
	\begin{itemize}\vspace{-0.5em}
		\item Developed software to estimate visibility of satellite given orbit simulation and thermal
		      constraints.
		\item Performed multi-parameter regression on ro-vibrational dataset so to characterize
		      starburst galaxy NGC253.
	\end{itemize}
\end{rSection}

\begin{rSection}{Publications and Personal Projects}

	\begin{itemize}
		\item van der Togt et al., \textit{[Re] Badder Seeds: Reproducing the Evaluation of Lexical
			      Methods for Bias Measurement}, 2022. ReScience C 8,
		      \#40. \href{https://doi.org/10.5281/zenodo.6574704}{10.5281/zenodo.6574704}.
		\item
		      {\bf \href{https://www.giuliostarace.com/projects/gptrue-or-false/}{GPTrue or False}}
		      Browser extension showing the likelihood that given body of text was generated by
		      \href{https://openai.com/blog/better-language-models/}{OpenAI's GPT-2}.
		\item
		      {\bf \href{https://github.com/thesofakillers/iclingo}{iclingo}}
		      Jupyter Kernel for \href{https://en.wikipedia.org/wiki/Answer_set_programming}{answer set
			      programming}  with \href{https://potassco.org/clingo/}{clingo}.
		\item
		      {\bf \href{https://github.com/thesofakillers/bLANS}{bLANS}} Tool for converting sane-LaTeX
		      to \href{https://ans.app/}{ANS}-flavoured LaTeX and viceversa.
	\end{itemize}
\end{rSection}


\begin{rSection}{Miscellaneous Qualities}

	\begin{tabular}{ @{} >{\bfseries}l @{\hspace{6ex}} l }
		Languages             & English (Fluent), Italian (Fluent), Spanish (Professional)										\\
		Programming Languages & Python, Javascript, SQL, C/C++, HTML/CSS																			\\
		Libraries             & NumPy, PyTorch, Pandas, HuggingFace, matplotlib, OpenCV, statsmodels, sklearn	\\
		Frameworks and Specs  & PyTorch Lightning, WandB, NodeJS (Express), GraphQL, React										\\
		Technologies          & Vim, Git, tmux, VSCode, ssh, \LaTeX, FFMPEG, Docker, SLURM 
	\end{tabular}

\end{rSection}
\begin{center}
	\begin{tabular}{ccc}
		\href{https://www.giuliostarace.com}{giuliostarace.com} & \href{https://github.com/thesofakillers}{github.com/thesofakillers} & \href{https://www.linkedin.com/in/giuliostarace/}{linkedin.com/in/giuliostarace}
	\end{tabular}
\end{center}

\end{document}
