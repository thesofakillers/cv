%%%%%%%%%%%%%%%%%%%%%%%%%%%%%%%%%%%%%%%%%
% Medium Length Professional CV
% LaTeX Template
% Version 2.0 (8/5/13)
%
% This template has been downloaded from:
% http://www.LaTeXTemplates.com
%
% Original author:
% Rishi Shah
%
% Important note:
% This template requires the resume.cls file to be in the same directory as the
% .tex file. The resume.cls file provides the resume style used for structuring the
% document.
%
%%%%%%%%%%%%%%%%%%%%%%%%%%%%%%%%%%%%%%%%%

%----------------------------------------------------------------------------------------
%	PACKAGES AND OTHER DOCUMENT CONFIGURATIONS
%----------------------------------------------------------------------------------------

\documentclass{resume} % Use the custom resume.cls style
\usepackage{hyperref}
\usepackage{enumitem}
\setlist[itemize]{noitemsep, topsep=0pt}
\usepackage[left=0.5in,top=0.4in,right=0.5in,bottom=0.4in]{geometry} % Document margins
\newcommand{\tab}[1]{\hspace{.2667\textwidth}\rlap{#1}}
\newcommand{\itab}[1]{\hspace{0em}\rlap{#1}}
\name{Giulio Starace} % Your name
\address{giulio.starace@gmail.com \\ \today}

\begin{document}

\begin{rSection}{Education}

	{\bf MSc Artificial Intelligence} \hfill \underline{Amsterdam, The Netherlands}
	\\\href{https://www.dur.ac.uk/}{\underline{University of Amsterdam}} \hfill {\em September 2021
	- 2023 (ongoing)}

	{\bf BSc Hons. in Natural Sciences (Physics \& Computer Science)} \hfill \underline{Durham, UK}
	\\\href{https://www.dur.ac.uk/}{\underline{Durham University}} \hfill {\em October 2016 - July 2019}
	\begin{itemize}\vspace{-0.5em}
		\item Natural Sciences degree allows candidates to take modules from selected departments
		\item BSc Project with \href{https://us.pg.com/}{\textit{P\&G}}: exploring methods for simulating wear-induced dry abrasion in clothes.
		\item Achieved First-Class Honours
	\end{itemize}
\end{rSection}

\begin{rSection}{Experience}

	{\bf \href{https://www.eso.org}{ESO} - European Southern Observatory} \hfill \underline{Cerro Paranal, Chile}
	\\ \underline{Research Engineer} \hfill {\em September 2020 - September 2021}
	\begin{itemize}\vspace{-0.5em}
		\item Worked on high-resolution forecasting of Astronomical Seeing using Machine Learning
			methods.
		\item Ran preliminary ARIMA-based approach before moving to LSTM neural networks.
		\item Developed Python PyPI library "nowcastlib", for more rigorous experimentation across colleagues.
	\end{itemize}
	\underline{Software Engineering Intern} \hfill {\em August 2019 - January 2020}
	\begin{itemize}\vspace{-0.5em}
		\item Part of the newly formed Info \& Knowledge Management Team, tasked with technical side.
		\item Designed, developed, tested and deployed a system for displaying information from the observatory's several ticket systems in a unified and personalized manner via Microsoft Graph API.
	\end{itemize}

	{\bf \href{https://video.io/}{Video.io - startup}} \hfill \underline{Remote, Anywhere}
	\\ \underline{ML Engineer} \hfill {\em February 2020 - June 2020}
	\begin{itemize}\vspace{-0.5em}
		\item Worked on implementing abstractive summarization of dialogic video. (Pandas, Google Cloud Speech API).
		\item Real time speech to text captioning of live video streams (NodeJS Streams, FFmpeg, HLS/DASH video)
		\item Revamping backend and devOps for internal experimental website (NodeJS, GraphQL, PostgreSQL, Docker, NGINX) with a little exposure to frontend (React)
	\end{itemize}

	{\bf \href{https://global.jaxa.jp/}{JAXA} -  Japan Aerospace Exploration Agency} \hfill \underline{Sagamihara, Japan}
	\\ \underline{Research Intern} \hfill {\em August 2018 - October 2018}
	\begin{itemize}\vspace{-0.5em}
		\item Part of LIRA (Laboratory for InfraRed Astrophysics)
		\item Developed software to estimate visibility of satellite given simulated orbits and thermal constraints
		\item Performed data-analysis, particularly multi-parameter regression, on ro-vibrational dataset in an attempt to characterize starburst galaxy NGC 253
	\end{itemize}
\end{rSection}

\begin{rSection}{Personal Projects}

	{\bf \href{https://chrome.google.com/webstore/search/Gptrue%20or%20false}{GPTrue or False}}  \hfill {\em November 2019}
	\begin{itemize}\vspace{-0.5em}
		\item Determine the likelihood that given body of text was generated by \href{https://openai.com/blog/better-language-models/}{OpenAI's GPT-2 model} in your browser.
		\item Featured on \href{https://www.digitaltrends.com/cool-tech/gpt-2-plugin-sorts-real-from-fake/}{digitaltrends.com}, \href{https://lifehacker.com/automatically-detect-computer-generated-text-with-this-1839942470}{lifehacker.com}, \href{https://techthelead.com/chrome-extension-tells-you-who-wrote-that-text-computer-or-human/}{techthelead.com} and \href{https://news.google.com/search?q=gptrue+or+false&hl=en-US&gl=US&ceid=US:en}{others}; 100+ stars, 10+ forks on \href{https://github.com/thesofakillers/GPTrue-or-False}{GitHub}.
		\item \href{https://twitter.com/BrendanEich/status/1196507724072095744?s=20}{Tweeted about} by Tesla AI director Andrej Karpathy, and Javascript creator Brendan Eich.
	\end{itemize}
\end{rSection}


\begin{rSection}{Miscellaneous Qualities}

	\begin{tabular}{ @{} >{\bfseries}l @{\hspace{6ex}} l }
		Languages             & English (Fluent), Italian (Fluent), Spanish (Professional)                   \\
		Programming Languages & Python, Javascript, SQL, C/C++, HTML/CSS                                     \\
		Frameworks and Specs  & NodeJS (Express), GraphQL, React, Protocol Buffers                     \\
		Libraries             & NumPy, Pandas, matplotlib, TensorFlow, PyTorch, OpenCV, statsmodels, sklearn \\
		Technologies          & Vim, Git, VSCode, ssh, Latex, FFMPEG, Docker
	\end{tabular}

\end{rSection}
\begin{center}
	\begin{tabular}{ccc}
		\href{https://www.giuliostarace.com}{giuliostarace.com} & \href{https://github.com/thesofakillers}{github.com/thesofakillers} & \href{https://www.linkedin.com/in/giuliostarace/}{linkedin.com/in/giuliostarace}
	\end{tabular}
\end{center}

\end{document}
